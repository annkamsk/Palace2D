\documentclass{article}
\usepackage{polski}
\usepackage[utf8]{inputenc}

\begin{document}

\title{Sprint 0}

\author{Mateusz Biegański, Anna Kramarska, Michał Sarzyński, Magda Suchodolska}
\maketitle

\section{Wizja}

Zamierzamy stworzyć dwuwymiarową grę zręcznościową, której głównym celem jest zbudowanie jak najwyższej wieży.
Klocki będą wysuwać się raz z prawej raz z lewej strony ekranu.
Jeżeli blok nie zostanie precyzyjnie postawiony na poprzednim, część wystajając poza podstawę odpadnie, a wielkość kolejnych zostanie pomniejszona o właśnie tę część.
Gdy gracz nie trafi klockiem w poprzedający, gra się kończy, a wynikiem gry jest liczba dobrze ustawionych bloków.

\section{Grupy użytkowników}
\begin{enumerate}
\item\textbf{Gracz:}
  \begin{itemize}
  \item Bierze udział w każdej rozgrywce
  \item Może sprawdzić 10 dotychczas najlepszych wyników
  \item Po każdej rozgrywce może wprowadzić uzyskany wynik i swój pseudonim do rankingu
  \end{itemize}
\end{enumerate}

\section{Funkcjonalność}
\begin{enumerate}
\item Uruchomienie aplikacji na komputerze z systemem Linux
\item Rozpoczęcie nowej gry i prowadzenie jej do momentu przegranej lub osiągnięcia maksymalnego wyniku
\item Zrestartowanie gry po zakończeniu poprzedniej
\item Przejście do rankingu najlepszych wyników
\item Wprowadzenie wyniku do rankingu po zakończeniu rozgrywki
\item Włączenie/wyłączenie melodii w tle
\end{enumerate}

\section{Harmonogram}
Planujemy następujący harmonogram:
\begin{itemize}
\item\textbf{Sprint 0} - 12.03 - Wizja, harmonogram, wybór technologii
\item\textbf{Sprint I} - 04.04 - Specyfikacja wymagań (wersja pierwsza), kod wykonywalny: plansza z jednym klockiem
\item\textbf{Sprint II} - 25.04 - Specyfikacja wymagań (dopracowanie), architektura systemu (wersja pierwsza), kod wykonywalny: budowanie wieży
\item\textbf{Sprint III} - 09.05 - Architektura systemu (dopracowanie), kod wykonywalny: dostosowanie okna
\item\textbf{Sprint IV} - 23.05 - Plakat (wersja pierwsza), kod wykonywalny: grafika
\item\textbf{Sprint V} - 30.05 - Plakat (dopracowanie), prezentacja, kod wykonywalny: tablica najlepszych wyników
\end{itemize}

\section{Technologie}
\begin{itemize}
\item\textbf{Język:} Java
\item\textbf{Framework:} libGDX
\end{itemize}


\section{Pozostałe narzędzia}
\begin{itemize}
\item\textbf{Repozytorium:} Github
\item\textbf{Testy jednostkowe:} JUnit
\end{itemize}


\end{document}
