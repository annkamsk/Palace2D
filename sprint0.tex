\documentclass{article}
\usepackage{polski}
\usepackage[utf8]{inputenc}

\begin{document}

\title{Sprint 0}

\author{Mateusz Biegański, Anna Kramarska, Michał Sarzyński, Magda Suchodolska}
\maketitle

\section{Wizja}

Zamierzamy stworzyć dwuwymiarową grę zręcznościową, której głównym celem jest zbudowanie jak najwyższej wieży.
Klocki będą wysuwać się raz z prawej raz z lewej strony ekranu.
Jeżeli blok nie zostanie precyzyjnie postawiony na poprzednim, część wystajając poza podstawę odpadnie, a wielkość kolejnych zostanie pomniejszona o właśnie tę część.
Gdy gracz nie trafi klockiem w poprzedający gra się kończy, a wynikiem gry jest ilość dobrze ustawionych bloków.

\section{Harmonogram}

12.03 - Sprint 0 - wizja + harmonogram
\\04.04 - Sprint I - Specyfikacja wymagań (wersja pierwsza), kod wykonywalny: plansza z jednym pionkiem
\\25.04 - Sprint II - Specyfikacja wymagań (dopracowanie), architektura systemu (wersja pierwsza), kod wykonywalny: budowanie wieży
\\09.05 - Sprint III - Architektura systemu (dopracowanie), kod wykonywalny: dostosowanie okna
\\23.05 - Sprint IV - Plakat (wersja pierwsza), kod wykonywalny: grafika
\\30.05 - Sprint V - Plakat (dopracowanie), prezentacja, kod wykonywalny: Best score

\section{Technologia}

\end{document}
