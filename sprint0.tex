\documentclass{article}
\usepackage{polski}
\usepackage[utf8]{inputenc}

\begin{document}

\title{Sprint 0}

\author{Mateusz Biegański, Anna Kramarska, Michał Sarzyński, Magda Suchodolska}
\maketitle

\section{Wizja}

Zamierzamy stworzyć dwuwymiarową grę zręcznościową, której głównym celem jest zbudowanie jak najwyższej wieży.
Klocki będą wysuwać się raz z prawej raz z lewej strony ekranu.
Jeżeli blok nie zostanie precyzyjnie postawiony na poprzednim, część wystajając poza podstawę odpadnie, a wielkość kolejnych zostanie pomniejszona o właśnie tę część.
Gdy gracz nie trafi klockiem w poprzedający, gra się kończy, a wynikiem gry jest liczba dobrze ustawionych bloków.

\section{Harmonogram}
Planujemy następujący harmonogram:
\begin{itemize}
\item\textbf{Sprint 0} - 12.03 - Wizja, harmonogram, wybór technologii
\item\textbf{Sprint I} - 04.04 - Specyfikacja wymagań (wersja pierwsza), kod wykonywalny: plansza z jednym pionkiem
\item\textbf{Sprint II} - 25.04 - Specyfikacja wymagań (dopracowanie), architektura systemu (wersja pierwsza), kod wykonywalny: budowanie wieży
\item\textbf{Sprint III} - 09.05 - Architektura systemu (dopracowanie), kod wykonywalny: dostosowanie okna
\item\textbf{Sprint IV} - 23.05 - Plakat (wersja pierwsza), kod wykonywalny: grafika
\item\textbf{Sprint V} - 30.05 - Plakat (dopracowanie), prezentacja, kod wykonywalny: Best score
\end{itemize}
\section{Technologia}

\end{document}
