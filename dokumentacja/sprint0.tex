
\documentclass{article}
\usepackage{polski}
\usepackage[utf8]{inputenc}

\begin{document}

\title{Palace 2D}

\author{Mateusz Biegański, Anna Kramarska, Michał Sarzyński, Magda Suchodolska}
\maketitle

\section{Wizja}

Zamierzamy stworzyć dwuwymiarową grę zręcznościową, której głównym celem jest zbudowanie jak najwyższej wieży.
Klocki będą wysuwać się raz z prawej raz z lewej strony ekranu.
Jeżeli blok nie zostanie precyzyjnie postawiony na poprzednim, część wystajając poza podstawę odpadnie, a wielkość kolejnych zostanie pomniejszona o właśnie tę część.
Gdy gracz nie trafi klockiem w poprzedający, gra się kończy, a wynikiem gry jest liczba dobrze ustawionych bloków.

\section{Grupy użytkowników}
\begin{enumerate}
\item\textbf{Gracz:}
  \begin{itemize}
  \item Bierze udział w każdej grze
  \item Może sprawdzić 10 dotychczas najlepszych wyników
  \item Po każdej grze może wprowadzić uzyskany wynik i swój pseudonim do rankingu
  \end{itemize}
\end{enumerate}

\section{Funkcjonalność}
\begin{enumerate}
\item Rozpoczęcie nowej gry i prowadzenie jej do momentu przegranej lub osiągnięcia maksymalnego wyniku
\item Wyświetlanie planszy z polem, na które mają spaść klocki
\item Sekwencyjne wysuwanie się klocków z prawej i lewej strony ekranu
\item Umieszczanie nowego klocka na szczycie wieży poprzez wciśnięcie spacji
\item Odcinanie części klocka, która nie zmieściła się w swoim polu docelowym
\item Przesuwanie widoku wieży w górę wraz z dokładaniem klocków
\item Zrestartowanie gry po zakończeniu poprzedniej
\item Przejście do rankingu najlepszych wyników
\item Wprowadzenie wyniku do rankingu po zakończeniu rozgrywki
\item Włączenie/wyłączenie melodii w tle
\end{enumerate}

\section{Harmonogram}
Planujemy następujący harmonogram:
\begin{itemize}
\item\textbf{Sprint 0} - 12.03 - Wizja, harmonogram, wybór technologii
\item\textbf{Sprint I} - 04.04 - Specyfikacja wymagań (wersja pierwsza), kod wykonywalny: rozpoczęcie nowej gry, plansza z jednym klockiem
\item\textbf{Sprint II} - 25.04 - Specyfikacja wymagań (dopracowanie), architektura systemu (wersja pierwsza), kod wykonywalny: budowanie wieży (lądowanie klocków, odcinanie), kończenie gry
\item\textbf{Sprint IV} - 23.05 - Plakat (wersja pierwsza), architektura systemu (dopracowanie), kod wykonywalny: 
  \begin{itemize}
    \item dynamiczne tworzenie klocków,
  \item możliwość budowania wielopoziomowej wieży (tzn. do poziomu określonego jako MAX),
  \item zmiana tekstur w zależności od aktualnej szerokości klocka (tzn. zarządzanie teksturami tak, aby ułożona wieża jak najbardziej przypominała Pałac),
  \item ekran kończący grę,
  \item wyświetlanie grafiki pełnej ułożonej przez gracza wieży po zakończeniu gry,
  \item ranking 10 najlepszych wyników,
  \item możliwość zapisu wyniku
  \end{itemize}
  \item\textbf{Sprint V} - 30.05 - Plakat (dopracowanie), grafika:
    \begin{itemize}
    \item dopracowanie wyglądu interfejsu (m.in dopracowanie wyglądu Pałacu),
      \item grafika ekranu rozpoczynającego grę,
    \item grafika MIM-u,
    \end{itemize}
    kod wykonywalny:
    \begin{itemize}
    \item ekran rozpoczynający grę,
    \item poruszanie się między ekranami,
    \item możliwość wyboru poziomu trudności gry przez gracza,
      \item nagradzanie gracza za dobrą grę (poszerzanie klocków, gdy w ciągu ostatnich 10 ruchów jego klocki zmniejszyły się o niewiele pikseli)
    \item możliwość wyboru skórki przez gracza (budowanie Pałacu czy MIM-u) - w zależności od tego wyboru interakcja z użytkownikiem może przebiegać nieco inaczej
    \item budowa MIM-u (lądowanie klocków, przesuwanie widoku)
    \item zapis wyniku dla MIM-u
    \end{itemize}
    \item\textbf{Prezentacja} - 05.06 - Prezentacja na zajęciach 
\end{itemize}

\section{Technologie}
\begin{itemize}
\item\textbf{System:} Linux
\item\textbf{Język:} Java
\item\textbf{Framework:} libGDX
\end{itemize}


\section{Pozostałe narzędzia}
\begin{itemize}
\item\textbf{Repozytorium:} Github
\item\textbf{Testy jednostkowe:} JUnit
\item\textbf{Grafika:} GIMP
\end{itemize}

\end{document}
